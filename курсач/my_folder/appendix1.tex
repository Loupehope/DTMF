\chapter{Исходный код}\label{SkinCancerClassifier}

\begin{lstlisting}[language=python]
	
# Таблицы частот для кодирования и декодирования

DTMF_TABLE = {
    '1': [1209, 697],
    '2': [1336, 697],
    '3': [1477, 697],
    'A': [1633, 697],

    '4': [1209, 770],
    '5': [1336, 770],
    '6': [1477, 770],
    'B': [1633, 770],

    '7': [1209, 852],
    '8': [1336, 852],
    '9': [1477, 852],
    'C': [1633, 852],

    '*': [1209, 941],
    '0': [1336, 941],
    '#': [1477, 941],
    'D': [1633, 941],
}

DTMF_FREQ = [1209.0, 1336.0, 1477.0, 1633.0, 697.0, 770.0, 852.0, 941.0]
DTMF_HIGH = [1209.0, 1336.0, 1477.0, 1633.0]
DTMF_LOW = [697.0, 770.0, 852.0, 941.0]

# -----------------
# Класс Reader
# -----------------

class Reader:

    @staticmethod
    def read(filename: str):
        # Читает wav файл.
        #
        # Parameters
        # ----------
        # filename : путь к файлу.
        #
        # Returns
        # -------
        # rate : int
        #     Частота дискретизации.
        # data : numpy array
        #     Данные файла.

        return scipy.io.wavfile.read(filename)

# -----------------
# Класс Writer
# -----------------

class Writer:

    @staticmethod
    def write(filename: str, signals: [Signal]):
        # Записывает данные в wav файл.
        #
        # Parameters
        # ----------
        # filename : путь к файлу.
        # signals : массив закодированных DTMF-сигналов.

        rate = 0
        data = np.array([])

        for signal in signals:
            rate = signal.rate
            data = np.append(data, signal.np_y_array())

        scipy.io.wavfile.write(filename, rate, data)
        
# -----------------
# Класс Signal
# -----------------
    
 class Signal:
    # Класс определеяет один DTMF-сигнал.
    #
    # Properties
    # ----------
    # data : путь к файлу.
    # symbol : символ DTMF-сигнала.
    # rate : частота дискретизации.

    # Initialization
    def __init__(self, data, symbol, rate):
        self.y_array = data[1]
        self.symbol = symbol
        self.rate = rate

    def np_y_array(self):
        # Преобразует данные в numpy array.
        #
        # Returns
        # -------
        # data : numpy array
        #     Данные файла.

        return np.array(self.y_array)
        	
# -----------------
# Класс Detector
# -----------------

class Detector:

    @staticmethod
    def detect(rate, data) -> str:
        # Декодирует wav файл.
        # 
        # Parameters
        # ----------
        # rate : путь к файлу.
        # data : данные сигнала.
        # 
        # Returns
        # -------
        # result : str
        #     декодированная строка сигнала.

        result = ""
        bin_size = int(rate * .25)
        goertzel = Goertzel(rate, bin_size)

        for i in range(0, len(data) - bin_size + 1, bin_size):
            goertzel.reset()

            for j in range(bin_size):
                goertzel.calc_s_n(data[i + j])

            powers = goertzel.calc_power()
            symbol = goertzel.get_number(powers)

            result += symbol

        return result
        
# -----------------
# Класс Generator
# -----------------
     
class Generator:

    @staticmethod
    def generate_from(symbols: str, duration=.25, volume=.25, rate=8000) -> [Signal]:
        # Кодирует строку в сигнал.
        #
        # Parameters
        # ----------
        # symbols : строка символов.
        # duration : длительность одного сигнала.
        # volume : громкость.
        # rate : частота дискретизации.
        #
        # Returns
        # -------
        # generated : [Signal]
        #     массив DTMF-сигналов.

        generated = []

        for symbol in symbols:
            generated.append(Generator.calculate(symbol, duration, volume, rate))

        return generated

    @staticmethod
    def calculate(symbol: str, duration=.25, volume=.25, rate=8000) -> Signal:
        # Кодирует символ в сигнал.
        #
        # Parameters
        # ----------
        # symbol : символ.
        # duration : длительность одного сигнала.
        # volume : громкость.
        # rate : частота дискретизации.
        #
        # Returns
        # -------
        # generated : Signal
        #     закодированный DTMF-сигналов.

        first_garmonik = GarmonikModel(volume, DTMF_TABLE[symbol.upper()][0], 0, int(duration * rate), 1, 1 / rate) \
            .trend(0, None)
        second_garmonik = GarmonikModel(volume, DTMF_TABLE[symbol.upper()][1], 0, int(duration * rate), 1, 1 / rate) \
            .trend(0, None)
        final_garmnonik = ModelDriver.add(first_garmonik, second_garmonik)

        return Signal(final_garmnonik, symbol, rate)
     
# -----------------
# Класс Goertzel
# -----------------
     
class Goertzel:
    # Реализация алгоритма Герцеля.

    def __init__(self, sample_rate: int, bin_size: int):
        self.s_prev = {}
        self.s_prev2 = {}
        self.coeff = {}

        for k in DTMF_FREQ:
            self.s_prev[k] = .0
            self.s_prev2[k] = .0

            freq_k = .5 + (bin_size * k) / sample_rate

            self.coeff[k] = 2.0 * math.cos(2.0 * math.pi * freq_k / bin_size)

    def get_number(self, powers):
        # Возвращает символ соответствующий
        # полученным частотных компонентам.
        #
        # Parameters
        # ----------
        # powers : магнитуды частот.
        #
        # Returns
        # -------
        # key : str
        #     декодированный DTMF-символ.

        high_freq = .0
        high_freq_temp = .0
        low_freq = .0
        low_freq_temp = .0

        for (high, low) in zip(DTMF_HIGH, DTMF_LOW):
            if powers[high] > high_freq_temp:
                high_freq_temp = powers[high]
                high_freq = high

            if powers[low] > low_freq_temp:
                low_freq_temp = powers[low]
                low_freq = low

        for key in DTMF_TABLE:
            if DTMF_TABLE[key][0] == high_freq and DTMF_TABLE[key][1] == low_freq:
                return key

    def calc_s_n(self, sample_data):
        # Вычисляет Sn.
        #
        # Parameters
        # ----------
        # sample_data : частота дикретизации.

        for freq in DTMF_FREQ:
            s = self.coeff[freq] * self.s_prev[freq] - self.s_prev2[freq] + sample_data
            self.s_prev2[freq] = self.s_prev[freq]
            self.s_prev[freq] = s

    def calc_power(self) -> {float: float}:
        # Вычисляет магнитуды частот.
        #
        # Returns
        # -------
        # powers : {float: float}
        #     словарь частот и их магнитуд.

        powers = {}

        for freq in DTMF_FREQ:
            power = self.s_prev2[freq] ** 2 + self.s_prev[freq] ** 2 - \
                    self.coeff[freq] * self.s_prev[freq] * self.s_prev2[freq]
            powers[freq] = power

        return powers

    def reset(self):
        # Удаляет ранее посчитанные данные.

        self.s_prev = {}
        self.s_prev2 = {}

        for k in DTMF_FREQ:
            self.s_prev[k] = .0
            self.s_prev2[k] = .0

\end{lstlisting}

%% В случае, когда таблица (рисунок) размещаются на последней странице, для переноса названия приложения на новую строку используем:
%\NewPage % начать новое приложение с новой страницы 
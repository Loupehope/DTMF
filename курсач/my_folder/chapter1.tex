\chapter{Теоретическая часть} \label{ch1}

\section{Формат входного сигнала}

В качестве входного сигнала был выбран аудиоформат wav по следующим причинам:
\begin{itemize}
	\item отсутствие сжатия данных;
	\item наличие готовых инструментов для чтения/записи данных.
\end{itemize}

% не рекомендуется использовать отдельную section <<введение>> после лета 2020 года
%\section{Введение. Сложносоставное название первого параграфа первой главы для~демонстрации переноса слов в содержании} \label{ch1:intro}

\section{Алгоритм генерации сигнала}

DTMF-сигнал представляет собой аддитивную модель двух гармонических процессов:
\begin{equation}
	x(t) = A_{0} * sin(2*\pi*n*f_{1}*\triangle{t}) + A_{0} * sin(2*\pi*n*f_{2}*\triangle{t}),
\end{equation}

где $A_{0}$ - амплитуда сигнала, $f_{1}$ и $f_{2}$ - частоты гармоник, $\triangle{t}$ - частота дискретизации.

Частоты гармоник берутся по приведённой ниже \taref{tab:lrt} из столбца и строки, соответствующих передаваемому символу. Каждая строка набора представлена частотой низкого тона, а каждый столбец - частотой высокого тона.

\begin{table}[ht]
\centering\small
	\caption{Таблица соответствия частот и символов DTMF}
	\label{tab:lrt}	
\begin{tabular}{|c|c|c|c|l|}
\hline
\multicolumn{1}{|l|}{\textbf{1209 Гц}} & \multicolumn{1}{l|}{\textbf{1336 Гц}} & \multicolumn{1}{l|}{\textbf{1477 Гц}} & \multicolumn{1}{l|}{\textbf{1633 Гц}} &                 \\ \hline
1                                      & 2                                     & 3                                     & A                                     & \textbf{697 Гц} \\ \hline
4                                      & 5                                     & 6                                     & B                                     & \textbf{770 Гц} \\ \hline
7                                      & 8                                     & 9                                     & C                                     & \textbf{852 Гц} \\ \hline
*                                      & 0                                     & \#                                    & D                                     & \textbf{941 Гц} \\ \hline
\end{tabular}\normalsize% возвращаем шрифт к нормальному
\end{table}

Процесс генерации сообщения двухтонального многочастотного набора заключается в последовательной генерации множества значений гармоник для каждого символа сообщения с последующей записью в файл.

\section{Алгоритм Гёрцеля}

Для решения задачи детектирования и декодирования тональных сигналов в телефонии обычно применяются две вариации дискретного преобразования Фурье: быстрое преобразование Фурье (FFT) и алгоритм Гёрцеля. В рамках данной работы разберем далее подробнее последнюю по причине того, что нам заранее известны частотные компоненты, которые мы хотим искать, что уменьшает необходимое количество подсчетов. 

Пусть $x_{n},\ n=0,\dots ,N-1$ — измеренные значения сигнала, которые являются входными данными для дискретного преобразования Фурье, а $X_{k},\ k=0,\dots ,N-1$ — частотные компоненты дискретного преобразования Фурье, по определению равные $X_{k}=\sum _{n=0}^{N-1}x_{n}e^{-{\frac {2\pi i}{N}}kn}$.
Для расчёта $X_{k}$ с помощью алгоритма Гёрцеля:
\begin{itemize}
	\item Последовательно вычисляются члены последовательности $s_n$ для $n=0, ..., N-1$ по рекуррентной формуле $s_{n}=2\cos \left({\frac {2\pi k}{N}}\right)s_{n-1}-s_{n-2}+x_{n}$, где $s_{-1}=s_{-2}=0$, $k = [0.5 + \frac{N * DTMF}{rate}]$, rate - частота дискретизации.
	\item Искомое значение частотного компонента получается как $X_{k}=e^{{\frac {2\pi i}{N}}k}s_{N-1}-s_{N-2}$.
\end{itemize}

Так как нам фаза сигнала не важна, на втором этапе алгоритма вместо комплексного значения частотного компонента вычислим квадрат его модуля по формуле: 

\begin{equation}
	|X_{k}|^{2}=s_{N-1}^{2}-2\cos \left({\frac {2\pi k}{N}}\right)s_{N-1}s_{N-2}+s_{N-2}^{2}.
\end{equation}

Следующим этапом выбираются две частоты: с самой большой мощностью и самой маленькой. После происходит поиск элемента в \taref{tab:lrt}, у которого совпадают значения высокой и низкой частоты с найденными ранее.

Чтобы найденный символ "засчитался", он должен  

\section{Выводы} \label{ch1:conclusion}



%% Вспомогательные команды - Additional commands
%
\newpage % принудительное начало с новой страницы, использовать только в конце раздела
%\clearpage % осуществляется пакетом <<placeins>> в пределах секций
%\newpage\leavevmode\thispagestyle{empty}\newpage % 100 % начало новой страницы
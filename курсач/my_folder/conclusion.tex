\chapter*{Заключение} \label{ch-conclusion}
\addcontentsline{toc}{chapter}{Заключение}	% в оглавление 

В ходе работы над курсовым проектом были выполнены следующие задачи:

\begin{itemize}
	\item изучены теоретические материалы по моделированию DTMF-сигналов;
	\item реализован метод генерации двухтонального многочастотного сигнала;
	\item изучены материалы по распознаванию DTMF-сигналов;
	\item выбран и реализован методов декодирования звукового сигнала.
\end{itemize} 

Можно подвести вывод, что процесс генерации исследуемых сигналов не отличается вариативностью и использовать что-либо иное отличное от простого сложения двух гармоник нецелесообразно. Разработанная модель позволяет записывать сообщения как с низкой частотой дискретизации, так и с довольно-таки высокой от 44100 и более, что является успехом.

Процесс декодирования DTMF-сигнала наоборот предлагает множество возможных решений. Выбранный алгоритм декодирования - алгоритм Гёрцеля - оказался эффективным и мощным инструментом для быстрого выделения необходимых частотных компонент в сообщении с низкой степенью ошибки даже при наличии шума. Таким образом, применение спектра Фурье для распознавания в специальной реализации показал себя с лучшей стороны.

Разработанное решение имеет пути дальнейшего развития и улучшения, например, можно декодирование сигнала можно обрабатывать меньшими пакетами для увеличения точности распознавания и уменьшения влияния шума.
 
Как итог, можно считать, что поставленная цель достигнута - создан инструмент для генерации DTMF-сигналов и распознавания их в звуковом файле соответственно.
